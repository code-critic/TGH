\documentclass[a4paper,10pt]{article}
%\usepackage[active]{srcltx}
\usepackage[czech]{babel}
\usepackage[utf8]{inputenc}

\usepackage{amsmath}
\usepackage{amsfonts}
\usepackage{amssymb}
\usepackage{amsthm}
\usepackage{hyperref}
%
\newtheorem{theorem}{Věta}[section]
\newtheorem{proposition}[theorem]{Tvrzení}
\newtheorem{definition}[theorem]{Definice}
\newtheorem{remark}[theorem]{Poznámka}
\newtheorem{lemma}[theorem]{Lemma}
\newtheorem{corollary}[theorem]{Důsledek}
\newtheorem{exercise}[theorem]{Cvičení}

%\numberwithin{equation}{document}
%
\def\div{{\rm div}}
\def\Lapl{\Delta}
\def\grad{\nabla}
\def\supp{{\rm supp}}
\def\dist{{\rm dist}}
%\def\chset{\mathbbm{1}}
\def\chset{1}
%
\def\Tr{{\rm Tr}}
\def\to{\rightarrow}
\def\weakto{\rightharpoonup}
\def\imbed{\hookrightarrow}
\def\cimbed{\subset\subset}
\def\range{{\mathcal R}}
\def\leprox{\lesssim}
\def\argdot{{\hspace{0.18em}\cdot\hspace{0.18em}}}
\def\Distr{{\mathcal D}}
\def\calK{{\mathcal K}}
\def\FromTo{|\rightarrow}
\def\convol{\star}
\def\impl{\Rightarrow}
\DeclareMathOperator*{\esslim}{esslim}
\DeclareMathOperator*{\esssup}{ess\,supp}
\DeclareMathOperator{\ess}{ess}
\DeclareMathOperator{\osc}{osc}
\DeclareMathOperator{\curl}{curl}
%
%\def\Ess{{\rm ess}}
%\def\Exp{{\rm exp}}
%\def\Implies{\Longrightarrow}
%\def\Equiv{\Longleftrightarrow}
% ****************************************** GENERAL MATH NOTATION
\def\Real{{\rm\bf R}}
\def\Rd{{{\rm\bf R}^{\rm 3}}}
\def\RN{{{\rm\bf R}^N}}
\def\D{{\mathbb D}}
\def\Nnum{{\mathbb N}}
\def\Measures{{\mathcal M}}
\def\d{\,{\rm d}}               % differential
\def\sdodt{\genfrac{}{}{}{1}{\rm d}{{\rm d}t}}
\def\dodt{\genfrac{}{}{}{}{\rm d}{{\rm d}t}}
\def\dx{\d x}
\def\Lin{\mathit{Lin}}
%
\def\vc#1{\mathbf{\boldsymbol{#1}}}     % vector
\def\tn#1{{\mathbb{#1}}}    % tensor
\def\abs#1{\lvert#1\rvert}
\def\Abs#1{\bigl\lvert#1\bigr\rvert}
\def\bigabs#1{\bigl\lvert#1\bigr\rvert}
\def\Bigabs#1{\Big\lvert#1\Big\rvert}
\def\ABS#1{\left\lvert#1\right\rvert}
\def\norm#1{\bigl\Vert#1\bigr\Vert} %norm
\def\close#1{\overline{#1}}
\def\inter#1{#1^\circ}
\def\eqdef{\mathrel{\mathop:}=}     % defining equivalence
\def\where{\,|\,}                    % "where" separator in set's defs
\def\timeD#1{\dot{\overline{{#1}}}}
%
% ******************************************* USEFULL MACROS
\def\RomanEnum{\renewcommand{\labelenumi}{\rm (\roman{enumi})}}   % enumerate by roman numbers
\def\rf#1{(\ref{#1})}                                             % ref. shortcut
\def\prtl{\partial}                                        % partial deriv.
\def\Names#1{{\scshape #1}}
\def\rem#1{{\parskip=0cm\par!! {\sl\small #1} !!}}
\def\reseni#1{\par{\bf Řešení:}#1}
%
%
% ******************************************* DOCUMENT NOTATIONS
% document specific
%***************************************************************************
%
\addtolength{\textwidth}{5cm}
\addtolength{\textheight}{4cm}
\addtolength{\topmargin}{-2cm}
\addtolength{\hoffset}{-2.5cm}
\begin{document}
\parskip=2ex
\parindent=0pt
\pagestyle{empty}
 \section*{Zadání příkladů pro semestrální práci 2019}

 Cílem seminární práce je aplikace teoretických znalostí z přednášky na konkrétní úlohy.
 Podstatu algoritmu totiž člověk nejlépe pochopí až pokud jej sám implementuje, nebo modifikuje při řešení konkrétního problému.
 
 \section{Termíny}
 \begin{itemize}
  \item {\bf 20.3.} končí možnost odevzdání podle loňského zadání pro opakující
  \item {\bf 17.3.} zveřejnění zadání pro 2020
  \item {\bf 27.3.} server \url{https://flowdb.nti.tul.cz/secure} aktualizován pro nové zadání
  \item do {\bf 21.4.} zaslání rozboru úlohy (textová část)
  \item do {\bf 22.5.} odevzdání první funkční verze programu
 \end{itemize}

 \section{Pravidla pro vypracování práce}

  \begin{itemize}
  \item   Studenti vytvoří řešitelské týmy po jednom nebo dvou lidech. Každý tým vypracuje řešení jedné úlohy dle vlastního výběru. Týmy nechť mi mailem nahlásí své složení a úlohu, kterou chtějí řešit.  Aktuální stav budu průběžně zveřejňovat v dokumentu:
 \href{https://drive.google.com/open?id=1eNz0321roZ7HAvuPMlfl2GOdHFszRVIkuvPEqD36uX0}{přehled týmů}
 
 Jednu úlohu může řešit maximálně 10 lidí v týmech celkem.

  \item Týmy vypracují textový rozbor problému. V rozboru formulujte úloha  pomocí grafu (co jsou vrcholy, co hrany, ohodnocení a pod.)
  a popište použití vhodného grafového algoritmu pro danou úlohu. Dále popište jaké datové struktury použijete. 
  Určete časovou a paměťovou složitost použitého algoritmu v nejhorším případě. 
  Textovou část ve formátu PDF mi zašlete na e-mail: jan.brezina@tul.cz 
         

  \item  Týmy implementují vybraný algoritmus. Pro implementaci můžete použít jeden z jazyků: C, C++, C\#, Java, Python, Pascal (viz. server)
   Program by měl být strukturovaný s výstižnými jmény identifikátorů, s okomentovanými funkcemi a hlavními datovými strukturami. 
   Program má skutečně mít časovou a paměťovou složitost uvedenou v textovém rozboru. Program čte vstup ze standardního vstupu a zapisuje výstup na standardní výstup.
  
  \item  Tým otestuje implementaci na testovacím serveru. Webové rozhraní váš program přeloží, 
        spustí na testovací sadě vstupů a porovná s referenčními výsledky. Pokud jste se svou prací spokojeni, nebo potřebujete jinou zpětnou vazbu, vyberte ``REQUEST CODE REVIEW''. Kód dále zkontroluji a přes seerver okomentuji, nebo potvrdím jeho přijetí.

  \item  Předchozí body plní tým jako celek, nicméně každý člen týmu musí rozumět jak teoretické části, tak programu a práce mu bude uznána, pokud v rámci zkoušky dokáže zodpovědět dotazy ohledně práce.
 \end{itemize}

\section{Zadání úloh}


\subsection{Přelévání nádob \hfill{\tt MINSPILL}}
 Máte k dispozici $n$ nádob s celočíselnými kladnými objemy $v_i<100$, $i\in 1,\dots n$ v litrech. 
 Největší nádoba o objemu $V$ je plná a cílem je odměřit daný počet litrů $w\le V$ v co nejkratším čase, přičemž přelévání probíhá vždy rychlostí $1$ litr za sekundu. Chceme získat postup optimálního přelévání pro všechny možné cílové objemy $w \le V$.
 Jedno přelití končí buď plnou cílovou nádobou nebo prázdnou vylévanou nádobou a nedochází ke ztrátám. 

 Formulujte problém jako grafovou úlohu. Co budou vrcholy, co hrany?  O jaký typ grafu se jedná?
 Navrhněte vhodné datové struktury umožňující běh Dijkstrova algortimu pro graf, který není dán na vstupu, ale vzniká dynamicky.
 Napište program používající vhodnou implementaci Dijkstrova algoritmu pro nalezení optimálních přelévacích postupů. 
 Pokuste se určit složitost vaší implementace v nejhorším případě: 
 \begin{enumerate}
  \item vzhledem k počtu vrcholů grafu 
  \item vzhledem k počtu nádob.
 \end{enumerate}


 Vstup je textový soubor, kde na prvním řádku je počet nádob, na dalších $n$ řádcích jsou objemy nádob.
 Na výstupu bude $V$ řádků. Na jednom výstupním řádku je postupně: cílový objem $w$, doba přelití (počet přelitých litrů) a počet přelití. Pokud objem $w$ nelze naměřit, vypíše se pouze $w$ následované koncem řádku.
 
Příklad vstupu:
\begin{verbatim}
3
2
4
6
\end{verbatim}

Očekávaný výstup:
\begin{verbatim}
1
2 2 1
3
4 2 1
5
6 0 0
\end{verbatim}


\subsection{IDOS \hfill{\tt kód IDOS}}
Uvažujte jednoduchý systém MHD s $n$ zastávkami a jízdními řády, které se každý den opakují. Jednotlivé spoje jsou dány
počáteční stanicí, cílovou stanicí, časem odjezdu (v minutách od začátku dne) a dobou cesty v minutách. Navrhněte algoritmus a napište program,
který pro zadaný čas, stratovní a cílovou stanici nalezne nejrychlejší spojení. Předpokládejte, že žadné spoje ani dotazy na spojení nejdou přes půlnoc.

Na prvním řádku vstupu je počet stanic $n$ a počet spojů $m$.
Na dalších $m$ řádkcích je pro každý spoj startovní a cílová stanice, čas odjezdu a 
doba cesty.  Na řádku $m+2$ je počet testovacích dotazů $N$ a na dalších $N$ řádcích jsou data pro jednotlivé dotazy. 
Jeden dotaz se skládá ze tří čísel: číslo startovní stanice, číslo cílové stanice a čas (v minutách od začátku dne) od kdy se má spojení hledat. 

Na výstupu bude $N$ řádků, pro každý dotaz jeden. Výsledek jednoho dotazu je posloupnost použitých spojů.
Jeden spoj je dán dvojicí: (číslo stanice, čas odjezdu). Dvojice jsou uzavřeny v závorce a odděleny mezerou. 
Po použitých spojích následuje ještě jedna dvojice obsahující index cílové stanice a čas příjezdu do ní.
Pokud pro daný dotaz spojení neexistuje, vypíše se prázdná závorka '()'.


Příklad vstupu:
\begin{verbatim}
4 6
0 1 10 30
0 1 70 30
0 2 130 10
1 2 60 30
1 3 100 60
3 1 30 20
2
0 2 0
0 3 20 
\end{verbatim}

Očekávaný výstup:
\begin{verbatim}
(0 10) (1 60) (2 90)
(0 70) (1 100) (3 160) 
\end{verbatim}



\subsection{Železnice \hfill{\tt kód: BIGLOKO1}}
Železniční společnost VELKÁ MAŠINA plánuje propojit železnicí $n$ měst, mezi nimiž zatím nikde železnice nevede. Byla vypracována studie, ve které bylo ohodnocena
výstavba železnice mezi některými dvojicemi měst. Pro některé dvojice měst bylo vypracováno více variant, celkem bylo uvažováno 
$m$ variant spojení mezi dvojicí některých měst. 
Studie také určila míru zátěže životního prostředí a obyvatelstva pro každou variantu a to stupni 1-5. 
Společnost VELKÁ MAŠINA potřebuje program, který pro zadané výsledky studie navrhne železniční síť, která propojí všechna města
a zároveň bude mít nejmenší zátěž životního prostředí. V případě více takových síťí bude vybrána síť nejlevnější.
Formulujte úlohu jako grafový problém hledání minimální kostry a navrhněte a implementujte algoritmus pro jeho řešení s časovou složitostí $O(m \log n)$ nebo lepší.

Vstupem je textový soubor, kde na prvním řádku jsou čísla $n$ a $m$. Na dalších $m$ řádcích jsou vždy data jedné varianty, tj. čtyři čísla. 
První dvě udávají indexy koncových měst úseku, tady jsou to čísla z množiny $\{0,\dots,n-1\}$, další číslo udává cenu vybudování úseku (v miliónech korun) 
opět celé číslo a poslední číslo udává míru zátěže životního prostředí z množiny ${1,2,3,4,5}$, pět je nejhorší. 

Na výstupu bude $n-1$ řádků s indexy variant železničních úseků (číslovaných od $0$ do $m-1$) použitých výslednou síť. Jejich pořadí je dáno následovně:
uvažujte výslednou kostru jako zakořeněný strom v městě $0$. Pro každé další město $j$ existuje jediný předek $k$. Na pozici $j-1$ v seznamu bude index $0$ až $m-1$ 
varianty použité pro spojení z $j$ do předka $k$. 


Příklad vstupu:
\begin{verbatim}
4 7
0 1 20  5
0 1 40  2
1 2 5   3
1 3 11  2
1 3 20  1
0 2 42  2 
2 3 30  4
\end{verbatim}


Očekávaný výstup:
\begin{verbatim}
1
5
4
\end{verbatim}


\subsection{Rozvrh \hfill{\tt kód: SCHED}}
Řešte zjednodušený problém sestavování rozvrhu. Jsou dány:
\begin{itemize}
 \item množina studentů $S=\{0,\dots, n_S - 1\}$, 
 \item množina předmětů $P=\{0,\dots, n_P - 1\}$,
 \item množina učitelů  $U=\{0,\dots, n_U - 1\}$, 
 \item množina výukových bloků $B=\{0,\dots, n_B - 1\}$. 
\end{itemize}
Každý student $i\in S$ má definovánu množinu zapsaných předmětů $p_i \subset P$. 
Každý předmět $j\in P$ má definovaného učitele $u_j \in U$. Předpokládáme neomezenou zásobu učeben.
Cílem je přiřadit předmětům vyučovací bloky tak, aby předměty které mají průnik studentů nebo učitelů
neměly stejný blok.
Úkoly:
\begin{enumerate} 
 \item Formulujte úlohu jako problém barvení grafu. Co jsou vrcholy, hrany, barvy?
 \item Vyberte si jeden z heuristických barvících algoritmů a popište jeho průběh při aplikaci na tento konkrétní problém.
 \item Zvojený algoritmus implementujte.
\end{enumerate}

Vstupem je textový soubor. Na prvním řádku jsou čísla $n_S$, $n_P$, $n_U$, $n_B$.
Na dalších $n_S$ řádcích je pro každého studenta $i$ seznam čísel jeho předmětů $p_i$. 
Na dalších $n_P$ řádcích je pro každý předmět $j$ číslo jeho učitele $u_j$.
Výstup programu má $n_P$ řádků na řádku $j$ je pro předmět $j$ číslo jeho bloku. 
Bloky jsou očíslovány tak, aby při ponechání pouze řádků s prvním výskytem bloku
tvořily uspořádanou posloupnost.

Příklad vstupu:
\begin{verbatim}
6 6 3 3
0 1 3
0 1 5
0 4 3
0 4 5
2 1 3
2 4 5
0
0
1
1
2
2
\end{verbatim}

Očekávaný výstup:
\begin{verbatim}
0
1
0
2
1
2
\end{verbatim}


\subsection{Sudoku \hfill{\tt kód: SUDOGOB}}
Ve vaší knize se Sudoku problémy řádil šotek, přidal nebo vymazal některá čísla ze zadání. Máte napsat program, který zjistí kolik 
řešení pro dané zadání existuje a pokud existuje právě jedno, tak ho vypíše.

Problém řešení sudoku formulujte jako grafovou úlohu (barvení grafu). Navrhněte a implementujte algoritmus, který pro zadané sudoku 9x9 nalezne 
jeho řešení. Použijte některý aproximativní algoritmus a doplňte ho o backtracking v případě, že by aproximativní řešení vyžadovalo více než 
9 barev.

Na prvním řádku vstupu je počet zadání $N$ k testování. Následuje $N$ tabulek oddělených prázdným řádkem. Jedna tabulka je $9$ řádků po devíti cifrách oddělených mezerou. 
Cifra $0$ označuje prázné pole. Výstupem je pro každou vstupní tabulku počet řešení, pokud má úloha jediné řešení, pak po počtu řešení následuje 
výpis unikátního řešení ve stejném formátu jako je vstup. Pokud úloha nemá řešení vypíše se počet řešení 0. 

\pagebreak
Příklad vstupu:
\begin{verbatim}
3
3 0 6 0 0 2 5 0 0
0 0 0 0 3 8 0 0 0
7 0 8 0 1 6 0 9 0
0 0 7 0 0 3 8 6 0
8 2 0 0 7 0 0 4 5
0 6 3 1 0 0 9 0 0
0 7 0 3 5 0 6 0 2
0 0 0 8 2 0 0 0 0
0 0 2 9 0 0 7 0 4

3 0 6 0 0 2 5 0 3
0 0 0 0 3 8 0 0 0
7 0 8 0 1 6 0 9 0
0 0 7 0 0 3 8 6 0
8 2 0 0 7 0 0 4 5
0 6 3 1 0 0 9 0 0
0 7 0 3 5 0 6 0 2
0 0 0 8 2 0 0 0 0
0 0 2 9 0 0 7 0 4

3 0 6 0 0 2 0 0 0
0 0 0 0 3 8 0 0 0
7 0 8 0 1 6 0 9 0
0 0 7 0 0 3 8 6 0
8 2 0 0 7 0 0 4 5
0 6 3 1 0 0 9 0 0
0 7 0 3 5 0 6 0 2
0 0 0 8 2 0 0 0 0
0 0 2 9 0 0 7 0 4
\end{verbatim}

Výstup:
\begin{verbatim}
1
3 1 6 4 9 2 5 7 8 
2 9 5 7 3 8 4 1 6 
7 4 8 5 1 6 2 9 3 
9 5 7 2 4 3 8 6 1 
8 2 1 6 7 9 3 4 5 
4 6 3 1 8 5 9 2 7 
1 7 9 3 5 4 6 8 2 
6 3 4 8 2 7 1 5 9 
5 8 2 9 6 1 7 3 4 
0
2
\end{verbatim}




\subsection{Dodávky elektřiny \hfill{\tt kód: DELIVERY}}
Model elektrické rozvodné sítě je tvořen elektrárnami $0$, \dots $n_e-1$  a rozvodnami $n_e$, \dots, $n_e + n_r-1$. Tyto \emph{přípojné body} jsou propojeny sítí čítající  $n_v$ vodičů. Jeden vodič vždy spojuje právě dva 
přípojné body a má definovaný maximální výkon, který jím může být přenášen (v obou směrech stejný). 
Žádné dvě elektrárny nejsou přímo propojeny a pro každou je znám její výkon. 
Elektrárny jsou napojeny libovolně na rozvodny. Rozvodny mohou být propojovány libovolně mezi sebou. 

Za účelem výstavby nového výrobního závodu bylo vytipováno $N$ míst. Pro každé místo $i=1,\dots ,N$ bylo navrženo $0 < k_i \le 3$ nezávislých elektrických přípojek napojených na přípojné body $r_{i,1}, \dots r_{i,k_i}$.
Navrhněte a implementujte algoritmus, který pro každou lokalitu určí maximální dostupný elektrický příkon.

{\bf Popis vstupu: }
Na prvním řádku vstupu jsou tři přirozená čísla udávající: počet elektráren $n_e$, počet rozvoden $n_r$  a počet vodičů $n_v$. Na dalších $n_e$ řádcích jsou výkony elekráren. Na dalších $n_v$ řádcích je specifikace vodičů. Na jednom řádku jsou vždy tři přirozená čísla udávající postupně: index prvního, index druhého propojeného přípojného bodu a maximální přenášený výkon.
Výkony elektráren a kapacity vodičů jsou přirozená čísla menší než 1000.
Následuje počet lokalit $N$ na samostatném řádku. Na každém z následujících $N$ řádcích 
je seznam přípojných bodů pro navržené elektrické přípojky (maximálně 3).
Na výstupu programu bude $N$ řádků, na každém maximální dostupný výkon pro odpovídající lokalitu.

Formulujte úlohu grafově jako úlohu pro tak, aby šla pro jednu lokalitu řešit jako problém sítě s jedním zdrojem a jedním cílem. Napište program pro nalezení optimálního toku sítí. Použijte Edmons-Karpův algoritmus (BFS pro hledání zlepšující cesty), nebo algoritmus s lepší složitostí.


Příklad vstupu:
\begin{verbatim}
2 3 4
4
4
0 4 6
1 4 5
4 2 4
4 3 7
3
2
3
2 3
\end{verbatim}

Očekávaný výstup:
\begin{verbatim}
4
7
8
\end{verbatim}



\end{document}

\subsection{Nejspolehlivější cesta \hfill{\tt kód: RELCONN}}
% aplikujeme -log( p ) a hledáme nejkratší cestu
Máme rozlehlou počítačovou síť, která je realizována
rádiovým spojením. Rádiové spojení může být nejrůznějším způsobem rušeno a tudíž
není moc spolehlivé. 
Síť tvoří uzly a jejich rádiová spojení. Uzly jsou očíslované $0$ až $n-1$ a jedno spojení je 
dáno: číslem $i$ vysílajícího uzlu, číslem $j$ přijímajícího uzlu a pravděpodobností $0<p_{i,j}<1$ přijmutí správného paketu.
Spojení mezi dvěma uzly sítě je vždy symetrické. 
Najděte v zadané síti nejspolehlivější cestu z vrcholu $s$ do vrcholu $t$. Nejspolehlivější
cesta je ta s nejmenší pravděpodobností chyby.

Na prvním řádku vstupu je počet uzlů $n$ a celkový počet spojení mezi nimi, $m$.
Na dalších $m$ řádkcích je pro každé spojení číslo vysílajícího uzlu, číslo  přijímajícího uzlu (číslováno od 0) a pravděpodobnost $p$ (např. 0.8931).
Na řádku $m+2$ je počet testovacích dotazů $N$ a na dalších $N$ řádcích jsou data pro jednotlivé dotazy. 
Jeden dotaz je dvojice uzlů, mezi kterými chceme najít cestu, první je index počátečního a druhý index koncového uzlu.

Na výstupu bude $N$ řádků, pro každý dotaz jeden. 
Výsledek jednoho dotazu je posloupnost indexů uzlů podél nalezené nejspolehlivější cesty včetně počátečního a koncového ze zadání dotazu.
V případě, že žádná cesta (byť s minimální spolehlivostí) pro dotaz neexistuje, vypíše se pouze index počátečního uzlu ze zadání dotazu.

Příklad vstupu:
\begin{verbatim}
3 3
0 1 0.8
0 2 0.5
1 2 0.7
1
0 2
\end{verbatim}

Očekávaný výstup:
\begin{verbatim}
0 1 2
\end{verbatim}


 

\subsection{Sudoku \hfill{\tt kód: SUDOGOB}}
Ve vaší knize se Sudoku problémy řádil šotek, přidal nebo vymazal některá čísla ze zadání. Máte napsat program, který zjistí kolik 
řešení pro dané zadání existuje a pokud existuje právě jedno, tak ho vypíše.

Problém řešení sudoku formulujte jako grafovou úlohu (barvení grafu). Navrhněte a implementujte algoritmus, který pro zadané sudoku 9x9 nalezne 
jeho řešení. Použijte některý aproximativní algoritmus a doplňte ho o backtracking v případě, že by aproximativní řešení vyžadovalo více než 
9 barev.

Na prvním řádku vstupu je počet zadání $N$ k testování. Následuje $N$ tabulek oddělených prázdným řádkem. Jedna tabulka je $9$ řádků po devíti cifrách oddělených mezerou. 
Cifra $0$ označuje prázné pole. Výstupem je pro každou vstupní tabulku počet řešení, pokud má úloha jediné řešení, pak po počtu řešení následuje 
výpis unikátního řešení ve stejném formátu jako je vstup. Pokud úloha nemá řešení vypíše se počet řešení 0. 

\pagebreak
Příklad vstupu:
\begin{verbatim}
3
3 0 6 0 0 2 5 0 0
0 0 0 0 3 8 0 0 0
7 0 8 0 1 6 0 9 0
0 0 7 0 0 3 8 6 0
8 2 0 0 7 0 0 4 5
0 6 3 1 0 0 9 0 0
0 7 0 3 5 0 6 0 2
0 0 0 8 2 0 0 0 0
0 0 2 9 0 0 7 0 4

3 0 6 0 0 2 5 0 3
0 0 0 0 3 8 0 0 0
7 0 8 0 1 6 0 9 0
0 0 7 0 0 3 8 6 0
8 2 0 0 7 0 0 4 5
0 6 3 1 0 0 9 0 0
0 7 0 3 5 0 6 0 2
0 0 0 8 2 0 0 0 0
0 0 2 9 0 0 7 0 4

3 0 6 0 0 2 0 0 0
0 0 0 0 3 8 0 0 0
7 0 8 0 1 6 0 9 0
0 0 7 0 0 3 8 6 0
8 2 0 0 7 0 0 4 5
0 6 3 1 0 0 9 0 0
0 7 0 3 5 0 6 0 2
0 0 0 8 2 0 0 0 0
0 0 2 9 0 0 7 0 4
\end{verbatim}

Výstup:
\begin{verbatim}
1
3 1 6 4 9 2 5 7 8 
2 9 5 7 3 8 4 1 6 
7 4 8 5 1 6 2 9 3 
9 5 7 2 4 3 8 6 1 
8 2 1 6 7 9 3 4 5 
4 6 3 1 8 5 9 2 7 
1 7 9 3 5 4 6 8 2 
6 3 4 8 2 7 1 5 9 
5 8 2 9 6 1 7 3 4 
0
2
\end{verbatim}





% Firmy si dluzi ruzne castky, urcete mnoziny firem. Firma X zproztredkovane dluzi firme Y pokud ...
% najdete mnoziny firem, ktere si vsechny navzajem zprostredkovane dluzi. 
%
% komplikace: redukce hran podle castky
% zprostredkovane dluhy podle castky
% 

\subsection{Segmentace \hfill{\tt kód: SEGMENT}}
Pro daný obrázek $n\times n$ pixelů ve stupních šedi $0 \dots 255$ je třeba odlišit pixely vyobrazeného objektu $O$ od 
kontrastního pozadí $B$ (problém segmentace obrazu). Přitom není známo zda jde o tmavý objekt na světlém pozadí nebo naopak,
ani není znám nějaký odstup intenzit objektu a pozadí. Předpokládáme však, že maximální rozdíl intenzit sousední pixelů $a$ a $b$ 
v rámci objektu ( $a, b \in O$) i v rámci pozadí ($a,b \in B$) je alespoň o $2$ stupně šedi menší než minimální rozdíl intenzit 
sousedních pixelů z nichž jeden je z objektu a druhý z pozadí $(a\in O,\ b\in B)$. Předpokládáme, že po obvodu obrázku 
jsou pouze pixely patřící k pozadí a detekovaný objekt nemá díry.

Navrhněte a popište algoritmus založený na hledání minimální kostry, kde ohodnocení hran je dáno rozdílem 
intenzit sousedících pixelů. Formulujte problém jako grafovou úlohu. Co budou vrcholy, co hrany?  
Určete složitost vaší implementace  vzhledem k počtu pixelů obrázku.

Vstup je textový soubor, kde na prvním řádku je velikost obrázku $n$ a na dalších $n$ řádcích je vždy $n$ čísel udavajících intenzity
v obrázku. Intenzity jsou čísla od 0 do 255. Výstup je ve stejném formátu jako vstup, jen místo intenzit je 
uvedena $0$ pro pixel pozadí a $1$ pro pixel objektu.

Příklad vstupu:
\begin{verbatim}
5
0 0 1 0 0
0 1 4 5 0
0 5 5 6 1
0 6 6 3 2
1 0 1 2 2 
\end{verbatim}

Očekávaný výstup:
\begin{verbatim}
0 0 0 0 0
0 0 1 1 0
0 1 1 1 0
0 1 1 0 0
0 0 0 0 0 
\end{verbatim}


%P-rozšíření: Jak by bylo možno segmentovat obraz na $N>2$ segmentů při zadaném minimální odstupu intenzit na hranicích segmentů? Implementujte tuto funkcionalitu.

%P-rozšíření:
%Jak by bylo možné segmentovat rozostřené obrázky, kde není ostrý skok intenzit na hranici segmenů, ale jen prudký gradient? (Zkuste vhodným způsobem přidat hrany do více okolních pixelů.)
%Implementujte tuto funkcionalitu.

%P-rozšíření:
%Analyzujte vhodnost použití Kruskalova a Borůvkova algoritmu v této konkrétní úloze, pokuste se přesněji odhadnout čas výpočtu a paměťovou náročnost.
%Své závěry ověřte praktickým porovnáním implementace pomocí Kruskalova a Borůvkova algoritmu.


\subsection{Minimal storage expression evaluation}
\begin{itemize}
\item Vstupem DAG vyrazu (umoznujeme promenne). Orientace hran od vysledku ke vstupu.
\item Chceme topologicke setrideni se specialni vlastnosti.
\item Pro pořadí $P$ a vrchol $v$ označíme $n_v^/$ počet vrcholů $u >= v$, do kterých
      vede hrana z nějakého vrcholu $w < v$.
\item Hledáme $/$ pro které je $\max_{v} n_v^P$ nejmenší.
\end{itemize}

Solution notes:
\begin{itemize}
\item Drawing vertices from the set $S$ (saturated vertices, zero indegree) in different order leads to different topological orderings.
\item We need a priority for the vertices in the set.
\item In the Kahn's algo. when we draw a vertex $v$ from $S$ and before we stack it to  $P$,  the set $S \setminus P$ 
      have size equal to $n_v^P$. Indeed, $S$ contains all vertices to which there is an edge from some $w < v$ and we subtract vertices
      before $v$ ($u<v$), that is the set $P$. In fact if we actually remove vertices from $S$ than $n_v^P$ is equalt to the size of $S$.
\item So the greedy variant of the Kahn's algorithm tries to  minimize maximal size of $S$, that is draw first vertices with minimum number $m_v$ of 
      neigbours that are not yet in $S$ (unvisited neighbours)
\item Simples implementation (with potentionaly quadratic complexity) just compute $m_v$ for all $v\in S$ and take the minimal vertex.
\item If we have also reverse edges available, than we can keep this number updated and use a priority queue.
\end{itemize}














\subsection{Plánovač pro Google Street View \hfill{\tt kód: GOSTRVI}}
Při pořizování dat pro službu Google  Street View, projíždí snímkovací vozidlo všechny ulice města. 
Město, které má vozidlo zmapovat je dáno očíslovanými křižovatkami $1\dots n$ a $m$ ulicemi. 
Jedna ulice je dána dvojicí křižovatek, které spojuje a  informací o jednosměrnosti.

Navrhněte a implementujte algoritmus, který:
\begin{enumerate}
\item Určí zda je město možné projet tak, aby se každá ulice projela právě jednou a vozidlo se vrátilo na místo startu. Jednosměrné ulice je třeba projet 
      ve správném směru.
\item Pokud platí podmínka 1), program nalezne optimální trasu.
\end{enumerate}
Návod: po určení vstupních a výstupních stupňů vrcholů je třeba určit orientaci neorientovaných hran tak, aby existovala Eulerovská kružnice. To lze například formulavat jako problém toku sítí 
z vrcholů s nedostatkem odchozích hran do vrcholů s přebytkem odchozích hran. Kapacity sítě jsou jednotkové. Po nalezení toku sítí budou dříve nevybalancované vrcholy vybalancované, ale mohou se objevit
nové nevybalancované vrcholy. Hledání toku je potřeba opakovat (na různých sítích) tak dlouho dokud nejsou všechny hrany zorientované, nebo nezjistíme, že je nelze zorientovat tak, aby byly všechny vrcholy vyvážené 
(v síti neexistuje tok).

Vstup je textový soubor, kde na prvním řádku jsou dvě čísla $n$ - počet křižovatek a $m$ - počet zadání ulic. 
Na dalších $m$ řádcích jsou zadány jednotlivé ulice. Jednosměrky jsou zadány dvojicí křižovatek: $i$ - počáteční křižovatka, $j$ - koncová křižovatka. 
Obousměrné ulice jsou zadány na dvou po sobě jdoucích řádcích jako dvě jednosměrné opačně orientované ulice.
Výstup je posloupnost křižovatek v navržené trase. Výstup má $k+1$ řádek, kde $k$ je počet všech ulic (jednosměrek i obousměrných) ve městě. 
První i poslední křižovatka na výstupu má index 0. Pokud existuje více Eulerovských kružnic, vypište tu která je první v lexikografickém uspořádání 
(křižovaky s nízkými indexy se projíždějí co nejdříve). Prázdný výstup v případě, že Eulerovská kružnice neexistuje.

\subsection{Inspekce vozovky \hfill{\tt kód: STREET_INSPECT}}
Pro monitorování vad na povrchu vozovky bylo vyvinuto vozidlo, které je schopno automaticky snímat a analyzovat jeden jízdní pruh.
Vaším cílem je napsat program, který pro danou silniční síť nalezne trasu inspekční vozidlo. 
Silniční síť je dána počtem křižovatek N a počtem silnic M, každá silnice je určena:
indexy dvou křižovatek A, B,  které spojuje a počtem pruhů pro směry A>B a B>A.

TODO: Upřesnit zadání tak aby pokud to lze nalezla se Eulerova kružnice. Pro testování nutno přijmout nejednoznačná řešení, tj. volat kontrolní program místo 
referenčních dat.

Při pořizování dat pro službu Google  Street View, projíždí snímkovací vozidlo všechny ulice města. 
Město, které má vozidlo zmapovat je dáno očíslovanými křižovatkami $1\dots n$ a $m$ ulicemi. 
Jedna ulice je dána dvojicí křižovatek, které spojuje a  informací o jednosměrnosti.

Navrhněte a implementujte algoritmus, který:
\begin{enumerate}
\item Určí zda je město možné projet tak, aby se každá ulice projela právě jednou a vozidlo se vrátilo na místo startu. Jednosměrné ulice je třeba projet 
      ve správném směru.
\item Pokud platí podmínka 1), program nalezne optimální trasu.
\end{enumerate}
Návod: po určení vstupních a výstupních stupňů vrcholů je třeba určit orientaci neorientovaných hran tak, aby existovala Eulerovská kružnice. To lze například formulavat jako problém toku sítí 
z vrcholů s nedostatkem odchozích hran do vrcholů s přebytkem odchozích hran. Kapacity sítě jsou jednotkové. Po nalezení toku sítí budou dříve nevybalancované vrcholy vybalancované, ale mohou se objevit
nové nevybalancované vrcholy. Hledání toku je potřeba opakovat (na různých sítích) tak dlouho dokud nejsou všechny hrany zorientované, nebo nezjistíme, že je nelze zorientovat tak, aby byly všechny vrcholy vyvážené 
(v síti neexistuje tok).

Vstup je textový soubor, kde na prvním řádku jsou dvě čísla $n$ - počet křižovatek a $m$ - počet zadání ulic. 
Na dalších $m$ řádcích jsou zadány jednotlivé ulice. Jednosměrky jsou zadány dvojicí křižovatek: $i$ - počáteční křižovatka, $j$ - koncová křižovatka. 
Obousměrné ulice jsou zadány na dvou po sobě jdoucích řádcích jako dvě jednosměrné opačně orientované ulice.
Výstup je posloupnost křižovatek v navržené trase. Výstup má $k+1$ řádek, kde $k$ je počet všech ulic (jednosměrek i obousměrných) ve městě. 
První i poslední křižovatka na výstupu má index 0. Pokud existuje více Eulerovských kružnic, vypište tu která je první v lexikografickém uspořádání 
(křižovaky s nízkými indexy se projíždějí co nejdříve). Prázdný výstup v případě, že Eulerovská kružnice neexistuje.




\subsection{Tube Challenge (P)}
Uvažujte stejný systém MHD jako v předchozí úloze a navrhněte a implementujte efektivní algoritmus pro nalezení nejrychlejší cesty
po všech stanicích  (viz. \url{http://en.wikipedia.org/wiki/Tube_Challenge})

Na prvním řádku vstupu je počet stanic $n$ a  počet spojů $m$
Na dalších $m$ řádkcích je pro každý spoj startovní a cílová stanice, čas odjezdu a 
doba cesty. Na výstupu bude pro každý použitý spoj jeden řádek obsahující: dobu čekání, čas odjezdu, startovní stanici, čas odjezdu a cílovou stanici

\subsection{Optimalizace robotů (P)}
Robot má provést $n$ operací (například vrtání) na $n$ různých místech výrobku (plošného spoje). Pro každou dvojici míst je známo, jak dlouho 
robotu trvá se mezi nimi přemístit, přitom cesta v opačných směrech může obecně trvat různě dlouho. 
Úkolem je najít optimální plán prováděn operací, aby se minimalizovala doba přemisťování.

Implmentujte vhodný aproximativní algoritmus pro řešení úlohy TSP. Specifikujte konkrétněji typ robota a pro něj stanovte
rozumné předpoklady o typu grafu a jeho ohodnocení. Existuje praktický připad, kdy by ohodnocení nebylo metrikou?

Vstupem je na prvním řádku číslo $n$ a na dalších $n^2$ řádcích jsou délky jednotlivých hran. Výstupem bude posloupnost vrcholů optimální 
cesty.





\subsection{Sekvencování DNA (Hamilton) (P)}
Při sekvencování DNA se tato nejprve rozdělí na menší části (300 -- 1000 bazí, každá pozice je obsazena jedním ze čtyř nukleotidů:
Adenin, Thymin, Guanin, Cytosin) a teprve tyto fragmenty se teprve přímo sekvencují. Výsledkem je velké množství řetězců náhodně vybraných 
z původní DNA, přičemž předpokládáme, že každé místo DNA je pokryto nějakým řetězcem (v praxi je každé místo pokryto minimálně třikrát).
Dalším zpracováním získáte pro každý řetězec $i$ seznam řetězců, které za něj lze připojit tak, aby se řetězce překrývali o nějakou minimální délku
a pro každou dvojici takto navazujících řeťezců $(i, j)$ máte délku maximálního překryvu řetězců 
(např. dvojice CAAA, AAG má maximální překryv délky 2). Cílem je naleznout posloupnost řeťezců tak, aby délka celkového spojeného 
řeťezce byla minimální a přitom se použily všechny zadané řetězce.
 
Formulujte úlohu jeko hlednání nejkratší Hamiltonovské cesty v grafu a implementujte některý z aproximativních algoritmů 
pro řešení tohoto problému. 


Vstupem programu bude textový soubor, kde na prvním řádku jsou čísla: $n$ počet řetězců, $m$ počet překryvů. Na dalších $m$ řádcích budou zadány jednotlivé překryvy:
$i$ index prvního řetězece, $j$ index druhého řetězce, $k$ pozice řetězce $j$ v řetězci $i$ (počítáno od nuly viz. předchozí příklad).
Výstupem bude posloupnost indexů řeťezců udávající nalezenou aproximaci optimálního řešení a d

\subsection{Kontaminace vody}
V jednom království mají nádhernou vodní zahradu spousta nádržek a rybníčků je propojena potůčky a fontánkami. Král se však obává teroristického útoku a chce vědět 
co se stane, pokud terorista otráví některou nádrž. Kam otrava dorazí za hodinu? Za jak dlouho otráví celou zahradu? A podobně.

 Systém, ve kterém proudí voda si budeme představovat jako soubor $n$ nádrží s objemy $V_i$, $i=1,\dots n$, které jsou propojeny
``hadičkami''. Každá hadička $h$ má dánu počáteční a koncovou nádrž a je dáno jakou rychlostí $q_h$ v ní voda proudí (v $m^3/s$). Předpokládáme, že systém je v rovnováze,
tedy do každé nádrže přitéká tolik kolik odtéká. Terorista nasype $m_I$ gramů jedu do nádrže $I$ v čase $0$. Simulaci šíření jedu si můžeme představit tak,
že pro každou nádržku určíme čas $t_i$:
\[
 t_i=V_i / Q_i,\quad Q_i = \sum_j q_{i,j}
\]
za jaký by z nádržky odtekla všechna voda, pokud by žádná nepřitékala ($Q_i$ je celkový odtok z nádržky $i$). Pak pro každou nádržku $i$ provádíme v pravidelných intervalech $t_i$
převod jedu ze zdrojové nádržky $i$ do cílové nádržky $j$ podle předpisů:
\[
    m_j = m_j + \Delta m_{i,j},\quad  \Delta m_{i,j} = q_{i,j} t_i c_i,\quad c_i = m_i/V_i,
\]
kde $c_i$ je koncentrace ve zdrojové nádržce a $\Delta m_{i,j}$ je množstí látky převáděné z nádržky $i$ do nádržky $j$. 
Po tomto převodu pak nastavíme $m_i=0$.

Navrhněte a implementujte algoritmus, který pro všechny nádržky určí za jaký čas se do nich jed dostane a v jakém množství (to je v případě, že krále zajímá kam se kdy jed dostane)
Implementaci proveďte tak, aby bylo možné ho ukončit v zadaném simulovaném čase $T$ (to je pro případ, kdy krále zajímá kam se jed dostane za jednu hodinu).

Vstupem je textový soubor, kde na prvním řádku je $n$ počet nádrží, $m$ počet hadiček, $I$ nádržka s jedem, $m_I$ množství jedu, $T$ maximální doba simulace.
Na dalších $n$ řádcích jsou objemy nádržek a na dalších $m$ řádcích jsou jednotlivé hadičky, tedy vždy zdrojová nádržka, cílová nádržka a rychlost proudění.

Výstupem je pro každou nádržku čas kontaminace ( může být i nekonečný pokud tam jed nestihne dorazit do času $T$) a množství jedu v čase první kontaminace.




Při stavění z lega je elementární oprerací spojení dvou dílů, nebo již částečně složených komponent. Máte dán přesný postup
sestavování cílového výrobku. Každému elementárnímu dílu i každé dílčí komponentě je přiřazeno jedno přirozené číslo. Pracovní 
postup je poslouponost trojic $(in1\ in2\ out)$, kde první dvě položky jsou vstupní komponenty a třetí je výstupní komponenta.
Předpokládejme, že každá elementární operace trvá 1s, dále máte k dispozici libovolný počet dobrovolníků pro skládání a 
předávání komponent je už zahrnuto v čase elementární operace. Vaším úkolem, je určit minimální čas $T$
potřebný k dokončení výrobku a pro každou operaci určit jak dlouho může maximálně trvat,
aby se celý výrobek stihnul v minimálním čase.

Formulujte problém jako grafovou úlohu. Co budou vrcholy, co hrany?  Navrhněte vhodný
algoritmus a určete jeho složitost vzhledem k počtu elementárních operací.

Implementujte algoritmus v jednom z jazyků: Pascal, C/C++, Java, Fortran.
Vstup je textový soubor, kde na prvním řádku je počet elementárních operací $n$, na dalších $n$ řádcích jsou
uspořádané trojice specifikující dvě vstupní a jednu výstupní komponentu. Na výstupu je $n$ čísel na $n$ řádcích
udávajících maximální dobu tvání pro každou operaci ve stejném pořadí jako na vstupu. Na řádku $n+1$ je minimální 
čas $T$.


\subsection{Robotický systém (SSK)}
Robotický systém má pracovní stavy $1,\dots, n$. Mezi stavy může přecházet pomocí svých motorických funkcí jako jsou
pohyby ramen, výměny nástrojů, přemisťování tekutin a pod. Přechody nejsou symetrické, tj. pokud lze přechod realizovat oběma směry, je třeba testovat každý směr zvlášť.
V rámci vývoje je třeba každý možný přechod otestovat předepsaným 
počtem provedení přechodu. Vývojový tým má k dispozici modul, který realizuje testování tak, že počítá počty realizovaných přechodů 
a z aktuálního stavu vždy provede přechod, pro který je rozdíl mezi požadovaným a realizovaným počtem přechodů největší. Modul ukončí testování, pokud
jsou pro všechny přechody dosaženy požadované počty opakování. 

\begin{itemize}
 \item Nalezněte příklad stavů, přechodů a požadovaných opakování jednotlivých přechodů tak, aby se testovací modul zacyklil.
 \item Nalzenout SSK a topologicky setřídit.... 
\end{itemize}
TODO: Upravit zadání. Prakticky použitelný robot musí mít graf přechodů DAG. Možná něco jako ekologický systém, kde mohou být jak cyklické závislosti, 
tak jednosměrné. Otázka je co by přesně měl graf reprezentovat a co by mělo být úkolem. 


\subsection{Robotický systém (Eulerův tah)}
Robotický systém má pracovní stavy $1,\dots, n$. Mezi stavy může přecházet pomocí svých motorických funkcí jako jsou
pohyby ramen, výměny nástrojů, přemisťování tekutin a pod. V rámci vývoje je třeba každý možný přechod otestovat předepsaným 
počtem provedení přechodu. Pro testování je třeba navrhnout trasu, která realizuje předepsané počty přechodů a minimalizuje 
počet nadbytečných přechodů.

Vaším úkolem je navrhnout algoritmus a napsat program, který určí zda je možno navrhnout trasu bez nadbytečných přechodů a
v takovém případě trasu najde. Určete složitost vaší implementace vzhledem k počtu přechodů. 

Zkuste se zamyslet nad použitelným řešením pro případ, kdy trasa bez nadbytečných přechodů neexistuje. Je tedy potřeba
minimalizovat nadbytečný počet přechodů za splnění podmínky pro existenci trasy.

Implementujte algoritmus v jednom z jazyků: Pascal, C/C++, Java, Fortran.
Vstup je textový soubor, kde na prvním řádku jsou dě čísla $n$ - počet stavů systému a $m$ - počet přechodů. 
Na dalších $m$ řádcích jsou popisy jednotlivých přechodů, tj, tři čísla $i$ - počáteční stav, $j$ - koncový stav, $k$ - předepsaný 
počet průchodů. Výstup je posloupnost stavů testovací trasy, nebo informace, že neexistuje trasa bez nadbytečných přechodů.





Při stavění z lega je elementární oprerací spojení dvou dílů, nebo již částečně složených komponent. Máte dán přesný postup
sestavování cílového výrobku. Každému elementárnímu dílu i každé dílčí komponentě je přiřazeno jedno přirozené číslo. Pracovní 
postup je poslouponost trojic $(in1\ in2\ out)$, kde první dvě položky jsou vstupní komponenty a třetí je výstupní komponenta.
Předpokládejme, že každá elementární operace trvá 1s, dále máte k dispozici libovolný počet dobrovolníků pro skládání a 
předávání komponent je už zahrnuto v čase elementární operace. Vaším úkolem, je určit minimální čas $T$
potřebný k dokončení výrobku a pro každou operaci určit jak dlouho může maximálně trvat,
aby se celý výrobek stihnul v minimálním čase.

Formulujte problém jako grafovou úlohu. Co budou vrcholy, co hrany?  Navrhněte vhodný
algoritmus a určete jeho složitost vzhledem k počtu elementárních operací.

Implementujte algoritmus v jednom z jazyků: Pascal, C/C++, Java, Fortran.
Vstup je textový soubor, kde na prvním řádku je počet elementárních operací $n$, na dalších $n$ řádcích jsou
uspořádané trojice specifikující dvě vstupní a jednu výstupní komponentu. Na výstupu je $n$ čísel na $n$ řádcích
udávajících maximální dobu tvání pro každou operaci ve stejném pořadí jako na vstupu. Na řádku $n+1$ je minimální 
čas $T$.



\subsection{Mobilní síť (barvení grafů) (T,P,P)}
Provozovatel mobilní telefonní sítě má $n$ vysílačů/přijímačů. Potřebuje jim přiřadit operační frekvence tak, 
aby se navzájem nerušily. Pro jednoduchost uvažujeme vysílače v rovině a s jednotným kruhovým dosahem 5km a minimální vzdáleností vysílačů 8km.

Formulujte problém jako grafovou úlohu. Co budou vrcholy, co hrany?  O jaký druh grafu se jedná. Jaký je optimální počet frekvencí?
Použijte vhodný aproximativní algoritmus pro přiřezení co nejmenšího počtu frakvencí. 

Vstup je textový soubor, kde na prvním řádku je počet vysílačů $n$ a na dalších $n$ řádcích jsou
jejich $x, y$ souřadnice v celých kilometrech. Výstup má $n$ řádků a na každém je číslo $1\dots ,k$ přiřazující 
vysílači příslušnou frekvenci. 

P-rozšíření: Navíc navrhněte a implementujte algoritmus, který bude rozmisťovat vysílače na dané ploše, tak aby pokrývali danou obdelníkovou oblast a 
co nejméně se rušili.

P-rozšíření: Upravte algortimus tak, aby vždy použil maximálně 5 frekvencí a napište funkci pro ověření rovinnosti grafu.





\subsection{Bludiště \hfill{\tt kód: MINOS}}

Krétský král Mínos plánuje stavět bludiště pro nevlastního krvelačného syna Minotaura. Bludiště má mít pouze jedno patro, bude kutáno v podzemí z jediného vchodu, chodby bludiště
budou pravoúhlé s celočíselnými délkami. Král chce, aby v bludišti existovala mezi každou dvojicí křižovatek právě jedna cesta.
Při kopání je však skála různě tvrdá (tvrdost $0$ až $2^{16}$) a stavitel chce aby se dělníci co nejméně nadřeli, přitom tvrdost skály je známá jen na stěnách již 
vykopaných chodeb. Chceme tedy bludiště, které má celkovou tvrdost vykopané horniny co nejmenší a pokud takových existuje více, 
chceme aby se při vynášení vykopané horniny urazila co nejmenší vzdélenost (součet vzdáleností všech křižovatek od vstupu).

Napište program na generování bludiště na čtvercové síti $n \times m$ bodů. Chodby mohou vést pouze po hranách čtvercové sítě. 
Kopání bludiště probíhá vždy pouze po jedné hraně a v okamžiku dosažení jejího koncového vrcholu se ohodnotí tvrdost dosud 
neohodnocených hran, které z něj vycházejí a to v pořadí (východ-0, západ-1, sever-2, jih-3). Tvrdost $\alpha_i$ se generujte náhodně pomocí
kongruenčního generátoru:
\[
    \alpha_{i+1} =  \alpha_i * 1664525 + 1013904223  \text{ mod } M
\]
hodnota $\alpha_0$, která je součástí vstupu, se použije pro ohodnocení první hrany. 
Hodnota $M$ je $2^{16}$.
Vstup do bludiště uvažujte vždy v levém horním rohu z levé strany.

Vstupem programu je jeden řádek se třemi celými čisly $n$, $m$, $\alpha_0$.
Výstupem bude pro každou křižovatku (bod sítě) je kód typu křižovatky. Z jedné křižovatky  existuje nebo neexistuje chodba 
do každého ze čtyř směrů (východ-0, západ-1, sever-2, jih-3), kód křižovatky je tedy 4-bitové číslo s uvedeným významem jednotlivých bitů. Jedna křižovatka je tedy
jedna cifra v 16-kové soustavě (např. křižovatka sever, západ jih je reprezentována cifrou $2+4+8=14=E$). 
Bludiště vypište na $n$ řádků po $m$ znacích.

Příklad vstupu:
\begin{verbatim}
3 3 3000
\end{verbatim}

Výsledné bludiště, zaokrouhlené ceny hran v tisících:
\begin{verbatim}
X -  3 - X -  1 - X
|                  
56       64       65
|         
X - 59 - X - 17 - X
                  |
61       58       46
                  |
X - 31 - X -  7 - X
\end{verbatim}

Výstup:
\begin{verbatim}
932
53A
136
\end{verbatim}
